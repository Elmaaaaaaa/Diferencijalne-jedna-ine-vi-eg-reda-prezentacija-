\documentclass{beamer}

\mode<presentation>{
%\usetheme{Warsaw}
%\usetheme{default}
%\usetheme{AnnArbor}
%\usetheme{Antibes}
%\usetheme{Bergen}
%\usetheme{Berkeley}
%\usetheme{Berlin}
%\usetheme{Boadilla}
%\usetheme{CambridgeUS}
%\usetheme{Copenhagen}
%\usetheme{Darmstadt}
%\usetheme{Dresden}
%\usetheme{Frankfurt}
%\usetheme{Goettingen}
%\usetheme{Hannover}
%\usetheme{Ilmenau}
%\usetheme{JuanLesPins}
%\usetheme{Luebeck}
\usetheme{Madrid}
%\usetheme{Malmoe}
%\usetheme{Marburg}
%\usetheme{Montpellier}
%\usetheme{PaloAlto}
%\usetheme{Pittsburgh}
%\usetheme{Rochester}
%\usetheme{Singapore}
%\usetheme{Szeged}
%\usetheme{Warsaw}
\usepackage{tcolorbox}
\usepackage{lipsum}
\newcommand{\arctg}[1]{\text{arctg}(#1) }
\newcommand{\tg}[1]{\text{tg}(#1)}
}
\usepackage[utf8]{inputenc}
\usepackage{default}
\usepackage[portuguese]{babel}
\usepackage{pgfplots}
\pgfplotsset{/pgf/number format/use comma,compat=newest}
\usepackage{color}
\usepackage{amsmath,amsfonts,amssymb}
\usepackage{hyperref}
\usepackage{tikz}
\usepackage{enumerate}
\usepackage{listings}

\newtheorem{deff}{Definicija}
\newtheorem{thm}{\textrm{Teorem}}[section]
\newtheorem{defn}{\textrm{Definicija}}[section]
\newtheorem{pos}{\textrm{Posljedica}}[section]
\newtheorem{lema}{\textrm{Lema}}[section]
\newtheorem{pri}{\textrm{Primjer}}[section]
\newtheorem{prim}{\textrm{Primjedba}}[section]
\newenvironment{dokaz}{\noindent\textbf{\textrm{Dokaz:}}}{\rule{0.3cm}{0.3cm}}
\newenvironment{rje}{\noindent\textit{Rje\v senje:}}{$\clubsuit$}
\renewcommand{\dj}{d\kern-0.4em\char"16\kern-0.1em}
\renewcommand{\DJ}{\raise0.3ex\hbox{-}\kern-0.36em D}
\renewcommand{\figurename}{Slika}








\title{DIFERENCIJALNE JEDNAČINE VI\v SEG  REDA}
\author{ \bf 
}
\institute[]{\includegraphics[height=1in]{index.png}{\vspace{.1cm}}\\ \bf UNIVERZITET U TUZLI {\vspace{.1cm}} \\  FAKULTET ELEKTROTEHNIKE {\vspace{.1cm}}\\ {\bf AKADEMSKA: 2019./20.}}
\date{\hspace{8cm} }


\begin{document}

\begin{frame}
 \maketitle 
\end{frame}

%\begin{frame}
%\frametitle{Sumário}
 %\tableofcontents
%\end{frame}


\section{Homogene diferencijalne jednačine višeg reda sa konstantnim koeficijentima}



\begin{frame}
\frametitle{Homogene diferencijalne jednačine višeg reda sa konstantnim koeficijentima} 
\begin{alertblock}{Opšti oblik}\begin{eqnarray}
y^{(n)} + a_{1}y^{(n-1)}+ ... + a_{n}y = 0
\end{eqnarray}
gdje su $a_{1},...,a_{n} \in \mathbb{R}$ poznate konstante. \end{alertblock}
\begin{alertblock}{Način rješavanja}Rje\v senje  diferencijalne jednačine (3) je oblika:  $$y = e^{r x},$$  ako je $r$¸ rje\v senje karakteristi\v cne jedna\v cine:
\begin{eqnarray}
r^{n} + a_{1}r^{n-1}+ ... + a_{n} = 0.
\end{eqnarray}
Prilikom rje\v savanja jedna\v cine (4) mogu da nastupe sljede\' ci slu\v cajevi:
\end{alertblock}
\end{frame}

\begin{frame}
\frametitle{Homogene diferencijalne jednačine višeg reda sa konstantnim koeficijentima} 
\begin{alertblock}{Način rješavanja}\begin{enumerate}
    \item Rješenje $r_i$ je jedinstveno i $r_i \in \mathbb{R}$. Tada je $y_{hi} = C_{i}e^{r_ix}.$
    \item  Rješenje $r_i$ je višestruko i red višestrukosti je $k$ (ima k istih rješenja) i  $r_i \in \mathbb{R}$. Tada je: $$y_{hi} = C_{i}e^{r_ix} + C_{i + 1}xe^{r_ix} +  C_{i + 2}x^{2}e^{r_ix} +...+ C_{i + k}x^{k-1}e^{r_ix}  .$$
    \item Rješenje $r_i$ je konjugovano kompleksni broj oblika: $\alpha \pm i\beta$. Onda je: $$y_{hi} = e^{\alpha x}(C_{i} \cos{\beta x} + C_{i+1} \sin{\beta x}).$$ 
\end{enumerate}
Rješenje polazne jednačine je: $y_{H}= \sum{y_{hi}}.$
\end{alertblock}
\end{frame}
\begin{frame}
\frametitle{Homogene diferencijalne jednačine višeg reda sa konstantnim koeficijentima} 

\begin{pri}
Odrediti opšte rješenje diferencijalne jednačine $$y''' + 3y'' + 3y' + y = 0,$$ te partikularno
 rješenje uz početne uslove: $y(0) = 1, $ $y'(0) = 2,$ i $y''(0) = 3$.
\end{pri}
\emph{Rješenje:}

Karakteristična jednačina : 
Uvodimo smjenu: $y= e^{rx}$ \\
$y'= re^{rx }, $ $y''= r^{2}e^{rx }, $ $y'''= r^{3}e^{rx }. $
$$r^{3}e^{rx } + 3r^{2}e^{rx } + 3re^{rx } + e^{rx } = 0 / : e^{rx }$$
$$r^{3} + 3r^{2} + 3r + 1 = 0$$

$$(r + 1)^{3} = 0$$ 
\end{frame}
\begin{frame}
\frametitle{Homogene diferencijalne jednačine višeg reda sa konstantnim koeficijentima} 
 
 $$r_{1,2,3} = -1.$$  
 
 Rješenje je višestruko, višestrukosti $k=3.$ 
 Opšte rješenje: 
 
$$y(x) = C_{1}e^{-x} + C_{2}xe^{-x} + C_{3}x^{2}e^{-x}.$$
Tražimo partikularno rješenje:

$$y(0) = 1.$$

$$1 = C_{1}e^{0} + C_{2}\cdot 0 \cdot e^{0}+ C_{3} \cdot0^{2} \cdot e^{0}.$$

$$ C_{1} = 1 .$$
\end{frame}

\begin{frame}
\frametitle{Homogene diferencijalne jednačine višeg reda sa konstantnim koeficijentima} 

Tražimo partikularno rješenje:



$$y'(0) = 2.$$

$$y'(x) =-C_{3}x^{2}e^{-x}+ 2C_{3}xe^{-x} - C_{2}xe^{-x}+ C_{2}e^{-x} -  C_{1}e^{-x}.$$

$$2 = -C_{3}\cdot 0^{2}\cdot e^{0}+ 2C_{3}\cdot 0 \cdot e^{0} - C_{2}\cdot 0 \cdot e^{0}+ C_{2}e^{0} -  C_{1}e^{0}.$$

$$2 =  C_{2} -  C_{1}.$$

$$2 =  C_{2} -  1.$$

$$C_{2} = 3.$$
\end{frame}
\begin{frame}
\frametitle{Homogene diferencijalne jednačine višeg reda sa konstantnim koeficijentima}
$$y''(0) = 3$$

$$y''(x) =C_{3}x^{2}e^{-x} - 4C_{3}xe^{-x} + C_{2}xe^{-x} + 2C_{3}e^{-x} - 2C_{2}e^{-x} +  C_{1}e^{-x}.$$
$$3 =C_{3} \cdot 0^{2} \cdot e^{0} - 4C_{3}\cdot 0 \cdot e^{0} + C_{2}\cdot 0 \cdot e^{0} + 2C_{3}e^{0} - 2C_{2}e^{0} +  C_{1}e^{0}.$$

$$3 = 2C_{3} - 2C_{2} +  C_{1}.$$

$$3 = 2C_{3} - 2 \cdot 3 +  1.$$
$$C_{3} = 4.$$
Partikularno rješenje :

$y(x) = e^{-x} + 3xe^{-x} + 4x^{2}e^{-x}.$

\hskip 10 cm $\clubsuit$ \\
\end{frame}

\begin{frame}
\frametitle{Homogene diferencijalne jednačine višeg reda sa konstantnim koeficijentima}

\begin{pri}
Odrediti opšte rješenje diferencijalne jednačine $$y^{(4)} + 2y'' -3y = 0.$$
\end{pri}

\emph{Rješenje:}

Karakteristična jednačina :

$$r^{4} + 2r^{2} -3 = 0,$$


$$(r^{2} - 1)(r^{2} + 3) = 0.$$

Rješenja k.j.:

$r_{1} = -1, $ $r_{2} = 1, $ $r_{3,4} = \pm \sqrt{3}i .$  \\

 $r_{1} = -1 $ jedinstveno i $r_{1} \in \mathbb{R}, $ to je $$y_{h_{1}} = C_{1}e^{-x}.$$ 
 

 
 \end{frame}
 
 \begin{frame}
\frametitle{Homogene diferencijalne jednačine višeg reda sa konstantnim koeficijentima}
$r_{2} = 1 $  je takođe jedinstveno i $r_{2} \in \mathbb{R}, $
 $$y_{h_{2}} = C_{2}e^{x}.$$
 
 Dalje, rješenje $r_{3,4} = \pm \sqrt{3}i $ je konjugovano kompleksni broj. Stoga je:
 
 $$y_{h3} = e^{0 \cdot x}[C_{3} \cos{(\sqrt{3} x)} + C_{4} \sin{(\sqrt{3} x)}].$$
 
 Odnosno:
 
 $$y_{h3} = C_{3} \cos{(\sqrt{3} x)} + C_{4} \sin{(\sqrt{3} x)}.$$
 
Opšte rješenje: $ y_{H}= y_{h1} + y_{h2} + y_{h3}.  $

 $$y_{H} =  C_{1}e^{-x} +  C_{2}e^{x} + C_{3} \cos{(\sqrt{3} x)} + C_{4} \sin{(\sqrt{3} x)}.$$
 
 \hskip 10 cm $\clubsuit$ \\
 

\end{frame}
\section{Nehomogene diferencijalne jednačine višeg reda sa konstantnim koeficijentima}



\begin{frame}
\frametitle{Nehomogene diferencijalne jednačine višeg reda sa konstantnim koeficijentima} 
\begin{alertblock}{Opšti oblik} \begin{eqnarray}
y^{(n)} + a_{1}y^{(n-1)}+ ... + a_{n}y = f(x)
\end{eqnarray}
gdje su $a_{1},...,a_{n} \in \mathbb{R}$ poznate konstante. \end{alertblock}
\begin{alertblock}{Način rješavanja}
Rješenje je oblika: $y= y_H + y_P$
Rješenje $y_H$ nalazimo iz odgovarajuće homogene jednačine (kao da umjesto $f(x)$ stoji 0).\\ Što se tiče nalaženja partikularnog rješenja $y_P$, imamo dva načina: metoda neodređenih koeficijenata i metoda varijacije konstanti.Mi ćemo ovdje koristiti metodu neodređenih koeficijenata! 
\subsubsection{Metoda neodređenih koeficijenata}
Ovdje podrazumijevamo samo specijalne slučajeve funkcije $f(x)!$
\end{alertblock}
\end{frame}

\begin{frame}
\frametitle{Nehomogene diferencijalne jednačine višeg reda sa konstantnim koeficijentima} 
\begin{alertblock}{Način rješavanja}
\begin{enumerate}
    \item[1.] $$f(x) = e^{\alpha x}P_{n}(x), $$ gdje je $P_{n}(x)= b_{0}x^{n} + b_{1}x^{n-1} +...+ b_{n} - $ polinom $n$ - tog stepena sa realnim koeficijentima i $\alpha \in \mathbb{R}.$
    \begin{enumerate}
        \item [(a)] Ako  $\alpha$ nije nula karakteristične jednačine, onda je: $y_{P}(x) = e^{\alpha x}R_{n}(x), $ pri čemu je: $R_{n}(x) = B_{0}x^{n} + B_{1}x^{n-1} +...+ B_{n}, $ gdje se koeficijenti $B_{0},...,B_{n}$ određuju smjenom $y_{P}, $ $y'_{P}, $ $y''_{P}, $ itd. u polaznu diferencijalnu jednačinu.
        \item [(b)] Ako je $\alpha$  nula karakteristične jednačine, onda je: $$y_{P}(x) = xe^{\alpha x}R_{n}(x), $$ 
        \item [(c)] Ako je $\alpha$  nula reda $k$ karakteristične jednačine, onda je: $$y_{P}(x) = x^{k}e^{\alpha x}R_{n}(x).$$ 
    \end{enumerate}
    \end{enumerate}
    \end{alertblock}
    \end{frame}
    \begin{frame}
\frametitle{Nehomogene diferencijalne jednačine višeg reda sa konstantnim koeficijentima} 
\begin{alertblock}{Način rješavanja}
\begin{enumerate}
    \item[2.] $$f(x) = e^{\alpha x}[P_{n}(x) \cos{\beta x} + Q_{m}(x) \sin{\beta x}],  $$ gdje je $P_{n}(x)$   polinom $n - $  tog stepena, a $Q_{m}(x)$ polinom $m - $  tog stepena s realnim koeficijentima i $\alpha, \beta \in \mathbb{R}.$
    \begin{enumerate}
        \item [(a)] $\alpha \pm i \beta$ nisu nule karakteristične jednačine, onda je:
        $$y_{P}(x) = e^{\alpha x}[R_{N}(x) \cos{\beta x} + S_{N}(x) \sin{\beta x}], $$ 
                  pri čemu su $R_{N}(x)$ i $S_{N}(x)$ polinomi $N - $  tog stepena, gdje je $N = \max(n,m)$ i  čiji se koeficijenti određuju smjenom $y_{P}, $ $y'_{P}, $ $y''_{P}, $ itd. u polaznu diferencijalnu jednačinu.
        
            \item [ (b)]$\alpha \pm i \beta$ su nule karakteristične jednačine, svaka reda $k, $ tada je:
            $$y_P(x) =  x^{k}e^{\alpha x}[R_{N}(x) \cos{\beta x} + S_{N}(x) \sin{\beta x}]. $$ 
        \end{enumerate}
        \end{enumerate}
        \end{alertblock}
        \end{frame}
         \begin{frame}
\frametitle{Nehomogene diferencijalne jednačine višeg reda sa konstantnim koeficijentima} 
\begin{alertblock}{Način rješavanja}
\begin{enumerate}
       
            \item[3.]  $$f(x) = h_{1}(x) + h_{2}(x) + ... +h_{m}(x) , $$ gdje su $h_{i}(x)$ $(i=1,...,m)$ oblika 1. ili 2. Tada je:
            $$y_P(x) = y_{P_{1}}(x) + y_{P_{2}}(x) +...+  y_{P_{m}}(x).$$
        \end{enumerate}
\end{alertblock}

\begin{pri}
Naći rješenje diferencijalne jednačine:

$$y'' + y' + y = x^2 + 3x +5.$$
\end{pri}

\end{frame}
  \begin{frame}
\frametitle{Nehomogene diferencijalne jednačine višeg reda sa konstantnim koeficijentima} 
\emph{Rješenje: }

$$y(x) = y_{H}(x) + y_{P} (x).$$ 

Tražimo rješenje pripadne homogene d.j. :

$$y'' + y' + y = 0. $$

Karakteristična jednačina:

$$r^{2} + r +1 = 0.$$

Rješenja K.J. su:\\

$r_{1} = -\frac{1}{2} +\frac{\sqrt{3}}{2}i $ , $ r_{2} = -\frac{1}{2} -\frac{\sqrt{3}}{2}i.$ \\

Rješenja $r_{1,2}$ su konjugovano kompleksni brojevi , gdje je:
$\alpha = -\frac{1}{2} $ i $ \beta = \pm \frac{\sqrt{3}}{2}. $

\end{frame}
\begin{frame}
\frametitle{Nehomogene diferencijalne jednačine višeg reda sa konstantnim koeficijentima} 

$$y_{H} = e^{-\frac{1}{2}x}\left[C_{1}\cos{\left(\frac{\sqrt{3}}{2} x \right) } + C_{2}\sin{\left(\frac{\sqrt{3}}{2} x\right)}\right].$$

Tražimo partikularno rješenje:

$$f(x) = x^2 + 3x +5, $$

$P_{n}(x)= x^2 + 3x +5 $ - polinom drugog stepena sa realnim koeficijentima i $\alpha = 0.$ \\

$$r_{1} = -\frac{1}{2} +\frac{\sqrt{3}}{2}i \ne \alpha, $$ 

 $$ r_{2} = -\frac{1}{2} -\frac{\sqrt{3}}{2}i \ne \alpha.$$
 
\end{frame}
\begin{frame}
\frametitle{Nehomogene diferencijalne jednačine višeg reda sa konstantnim koeficijentima} 

Partikularno rješenje je oblika:

$y_{P}(x) = e^{\alpha x}R_{n}(x). $

$$R_{n}(x) = Ax^{2} + Bx + C.  $$

$$y_{P}(x) =  e^{0 \cdot x}(Ax^{2} + Bx + C) = Ax^{2} + Bx + C. $$

$$y'_{P}(x) = 2Ax + B, $$

$$y''_{P}(x) = 2A .$$

$$y'' + y' + y = x^2 + 3x +5. \Longleftrightarrow$$

$$2A + 2Ax + B  + Ax^{2} + Bx + C = x^2 + 3x +5 .$$

$$A = 1$$
$$2A + B = 3$$

$$2A + B + C = 5$$
\end{frame}
\begin{frame}
\frametitle{Nehomogene diferencijalne jednačine višeg reda sa konstantnim koeficijentima} 
$$A = 1$$
$$2A + B = 3$$

$$2A + B + C = 5$$

$A = 1, $ $B = 1 $ i $C = 2.$ \\

$$ y_{P } = Ax^{2} + Bx + C = x^{2} + x + 2. $$

$$y_{H} + y_{P} =  e^{-\frac{1}{2}x}\left[C_{1}\cos{\left(\frac{\sqrt{3}}{2} x \right) } + C_{2}\sin{\left(\frac{\sqrt{3}}{2} x\right)}\right] + x^{2} + x + 2. $$

 \hskip 10 cm $\clubsuit$ \\
 
 
 
\end{frame}
\begin{frame}
\frametitle{Nehomogene diferencijalne jednačine višeg reda sa konstantnim koeficijentima}
 \begin{pri}
 Riješiti diferencijalnu jednačinu:
$$y^{(4)} - 6y''' + 9y'' = 3x^{2} + 2x.$$
 \end{pri}
\emph{Rješenje:}


Prvo rješavamo pripadnu homogenu, odnosno:

$$y^{(4)} - 6y''' + 9y'' = 0, $$

Karakteristične jednačine:

$$r^{4} - 6r^{3} + 9r^{2} = 0$$

$$r^{2} (r^{2} - 6r + 9) = 0$$



\end{frame}
\begin{frame}
\frametitle{Nehomogene diferencijalne jednačine višeg reda sa konstantnim koeficijentima} 
$$r^{2} (r-3)^{2} = 0$$

Dakle, rješenja su: $r_{1,2} = 0$ i $r_{3,4} = 3.$\\

 Dva višestruka rješenja i red višestrukosti je $k=2$ za oba.\\
 
 Za  $r_{1,2} = 0$  vrijedi:
 
$$y_{h1} = C_{1}e^{0 \cdot x} + C_{2}xe^{0\cdot x}  .$$

Za $r_{3,4} = 3$  vrijedi:

$$y_{h2} = C_{3}e^{3 \cdot x} + C_{4}xe^{3\cdot x}  .$$
Rješenje pripadne homogene jednačine je, dakle, zbir ova dva rješenja:

$$y_{H} = C_{1}e^{0 \cdot x} + C_{2}xe^{0\cdot x} + C_{3}e^{3 \cdot x} + C_{4}xe^{3\cdot x} = C_{1} + C_{2}x + C_{3}e^{3 \cdot x} + C_{4}xe^{3\cdot x}  .$$

Tražimo partikularno rješenje:

 

\end{frame}
\begin{frame}
\frametitle{Nehomogene diferencijalne jednačine višeg reda sa konstantnim koeficijentima} 
$$f(x) = 3x^{2} + 2x, $$

$P_{n}(x)= 3x^{2} + 2x  $ - polinom drugog stepena sa realnim koeficijentima i $\alpha = 0.$ 

$$r_{1,2} = 0 = \alpha, $$

tj. $ \alpha $ je nula reda $k=2.$\\

Partikularno rješenje je oblika:

$$y_{P}(x) = x^{k}e^{\alpha x}R_{n}(x).$$ 
Kako je $\alpha = 0, $ $k = 2$ i   $R_{n}(x) = Ax^{2} + Bx + C $\\

$$y_{P}(x) = x^{2}(Ax^{2} + Bx + C).$$


 

\end{frame}
\begin{frame}
\frametitle{Nehomogene diferencijalne jednačine višeg reda sa konstantnim koeficijentima} 
$$y'_{P}(x) = 4Ax^{3} + 3Bx^{2} + 2Cx$$

 $$y''_{P}(x) = 12Ax^{2} + 6Bx + 2C$$

 $$y'''_{P}(x) = 24Ax + 6B$$
 
 $$y^{(4)}_{P}(x) = 24A.$$
 
 $$y^{(4)} - 6y''' + 9y'' = 3x^{2} + 2x \Longleftrightarrow$$
 
 $$24A - 6( 24Ax + 6B)+ 9(12Ax^{2} + 6Bx + 2C) = 3x^{2} + 2x.$$
 
 $A = \frac{1}{36}, $ $B = \frac{2}{27}$  i $C = \frac{1}{9}.$


 




\end{frame}
\begin{frame}
\frametitle{Nehomogene diferencijalne jednačine višeg reda sa konstantnim koeficijentima}
$$y_{P} = \frac{1}{36} x^{4} + \frac{2}{27}x^{3} + \frac{1}{9}x^{2}.$$

Konačno rješenje:

$$y = y_{H} + y_{P} = C_{1} + C_{2}x + C_{3}e^{3 \cdot x} + C_{4}xe^{3\cdot x} + \frac{1}{36} x^{4} + \frac{2}{27}x^{3} + \frac{1}{9}x^{2}.$$

 \hskip 10 cm $\clubsuit$ \\
 
 \begin{pri}
 Riješiti diferencijalnu jednačinu:
$$y'' - 2y' = e^{-x}(x\cos{x} + 3 \sin{x}).$$
 \end{pri}

\emph{Rješenje: }

Rješavamo pripadnu homogenu:





\end{frame}

\begin{frame}
\frametitle{Nehomogene diferencijalne jednačine višeg reda sa konstantnim koeficijentima}
$$y'' - 2y' = 0. $$

 Karakteristična jednačina:
 
$$r^{2} - 2r = 0$$

$$r(r - 2) = 0$$

Dakle, vrijedi da je: $r_{1}= 0$ i $r_{2}= 2.$\\

Stoga je: $$y_{H} = C_{1} + C_{2}e^{2x}.$$

Tražimo partikularno rješenje:\\

$$f(x) = e^{-x}(x\cos{x} + 3 \sin{x})$$

U našem slučaju je $\alpha = -1$ i $\beta = 1, $ pa $\alpha + i \beta = -1 + i $ nije  nula k.j.


\end{frame}
\begin{frame}
\frametitle{Nehomogene diferencijalne jednačine višeg reda sa konstantnim koeficijentima}
$y_{P}$ je oblika: \\

$$y_{P}(x) = e^{\alpha x}[R_{N}(x) \cos{\beta x} + S_{N}(x) \sin{\beta x}], $$

$P_{n}(x) = x, $ tj.$P_{n}(x) $  je polinom prvog stepena,  a $Q_{m}(x) = 3$, tj.$Q_{m}(x) $ je polinom nultog stepena.
$R_{N}(x)$  i $S_{N}(x)$ - polinomi prvog stepena,  pa je:

$$R_{N}(x) = Ax + B$$ 

$$S_{N}(x) = Cx +D.$$ 

$$y_{P}(x) = e^{- x}[(Ax + B) \cos{x} + (Cx +D) \sin{ x}] .$$

$$y'_{P}(x) = e^{- x}[(Cx - Ax + A + D -B) \cos{x} + (C - B - Ax -Cx -D) \sin{ x}] .$$ 



\end{frame}
\begin{frame}
\frametitle{Nehomogene diferencijalne jednačine višeg reda sa konstantnim koeficijentima}
$$y''_{P}(x) = e^{- x}[(-2Cx - 2A + 2C - 2D) \cos{x} + (2Ax -2A + 2B - 2C) \sin{ x}] .$$ 


$$y'' - 2y' = e^{-x}(x\cos{x} + 3 \sin{x}). \Longleftrightarrow$$

$e^{- x}[(-2Cx - 2A + 2C - 2D) \cos{x} + (2Ax -2A + 2B - 2C) \sin{ x}] - 2e^{- x}[(Cx - Ax + A + D -B) \cos{x} + (C - B - Ax -Cx -D) \sin{ x}] = e^{-x}(x\cos{x} + 3 \sin{x}).$ \\

Odakle dobijamo da je:\\

$A = \frac{1}{10}, $ $B = -\frac{14}{25}, $ $C = -\frac{1}{5} $ i $D = \frac{2}{25}. $ \\

$$y_{P}(x) = e^{- x} \left[ \left(\frac{1}{10}x -\frac{14}{25}\right) \cos{x} + \left( -\frac{1}{5}x + \frac{2}{25}\right) \sin{ x}\right] .$$

$$y = C_{1} + C_{2}e^{2x} + e^{- x} \left[ \left(\frac{1}{10}x -\frac{14}{25}\right) \cos{x} + \left( -\frac{1}{5}x + \frac{2}{25}\right) \sin{ x}\right] .$$
\hskip 10 cm $\clubsuit$ \\

\end{frame}
\begin{frame}
\frametitle{Nehomogene diferencijalne jednačine višeg reda sa konstantnim koeficijentima}

\begin{pri}
Naći rješenje diferencijalne jednačine:
$$y''' - 4y' = xe^{2x} + \sin{x} + x.$$
\end{pri}

\emph{Rješenje: }

Rješavamo pripadnu homogenu:

$$y''' - 4y' = 0.$$

Karakteristična jednačina :

$$r^{3} - 4r = 0.$$

$$r(r^{2} - 4) = 0.$$

Rješenja karakteristične jednačine su:
\end{frame}
\begin{frame}
\frametitle{Nehomogene diferencijalne jednačine višeg reda sa konstantnim koeficijentima}

$r_{1} = 0 ,$  $r_{2} =  2 $ i $r_{3} =  -2 .$\\

Svi su  korijeni (rješenja) K.J. su različiti i realni, pa je: 

\begin{eqnarray}
y_{H } = C_{1} + C_{2}e^{2x} + C_{3}e^{-2x}. 
\end{eqnarray}
Tražimo partikularno rješenje:

$$f(x) = h_{1}(x) + h_{2}(x) + h_{3}(x)  = xe^{2x} + \sin{x} + x.$$

Partikularno rješenje je oblika:

\begin{eqnarray}
y_{P} = y_{P_{1}} + y_{P_{2}} + y_{P_{3}}.
\end{eqnarray}
\end{frame}
\begin{frame}
\frametitle{Nehomogene diferencijalne jednačine višeg reda sa konstantnim koeficijentima}



\begin{enumerate}
\item[$\star$] Za $h_{1}(x) = xe^{2x}$:\\

$\alpha = 2$ je nula k.j.:

$$y_{P_{1}} = x(Ax + B)e^{2x}.$$

$$y'_{P_{1}}(x) = e^{2x}(2Ax^{2}+ 2Bx + 2Ax +B) .$$ 

$$y''_{P_{1}}(x) = e^{2x}(4Ax^{2}+(4B+8A)x+4B+2A) .$$ 

$$y'''_{P_{1}}(x) =e^{2x} (8Ax^{2}+(8B+24A)x+12B+12A) .$$ 

\end{enumerate}
\end{frame}
\begin{frame}
\frametitle{Nehomogene diferencijalne jednačine višeg reda sa konstantnim koeficijentima}
$$y''' - 4y' = xe^{2x} \Longleftrightarrow$$

$$e^{2x} (8Ax^{2}+(8B+24A)x+12B+12A) - 4e^{2x}(2Ax^{2}+ 2Bx + 2Ax +B) = xe^{2x}.$$

$A = \frac{1}{16}$ i $B =- \frac{3}{32}.$\\

$$y_{P_{1}} = \left(\frac{1}{16}x^{2} -\frac{3}{32}x \right)e^{2x}.$$

\begin{enumerate}
    \item [$\star$] Za $ h_{2}(x) =\sin{x}:$\\
    
    $\pm i$ nije nula karakteristične jednačine:
    
$$y_{P_{2}} = C \cos{x} + D \sin{x}.$$

$$y'_{P_{2}}(x) = -C \sin{x}+ D \cos{x} .$$ 
\end{enumerate}
\end{frame}
\begin{frame}
\frametitle{Nehomogene diferencijalne jednačine višeg reda sa konstantnim koeficijentima}

$$y''_{P_{2}}(x) = -C \cos{x} -D \sin{x}  .$$ 

$$y'''_{P_{2}}(x) = C \sin{x} - D \cos{x} .$$ 
$$y''' - 4y' =  \sin{x}\Longleftrightarrow$$

$$C \sin{x} - D \cos{x} + 4C \sin{x} -  4D \cos{x} = \sin{x}$$

Stoga je: $C = \frac{1}{5}, $ $D = 0. $


$$y_{P_{2}} = \frac{1}{5} \cos{x}.$$

\end{frame}
\begin{frame}
\frametitle{Nehomogene diferencijalne jednačine višeg reda sa konstantnim koeficijentima}




\begin{enumerate}
\item[$\star$] Za $h_{3}(x) = x$:\\

$\alpha = 0$ rješenje K.J. :

$$y_{P_{3}} = x(Ex + F).$$

$$y'_{P_{3}}(x) = 2Ex + F .$$ 

$$y''_{P_{3}}(x) = 2E .$$ 

$$y'''_{P_{3}}(x) = 0 .$$ 

$$y''' - 4y' = x \Longleftrightarrow$$


\end{enumerate}
\end{frame}
\begin{frame}
\frametitle{Nehomogene diferencijalne jednačine višeg reda sa konstantnim koeficijentima}


$$0 - 8Ex - 4F = x.$$

$E = -\frac{1}{8}$ i $F =0.$

$$y_{P_{3}} = -\frac{1}{8}x^{2}.$$


$$y = y_{H} + y_{P} = y_{H} + (y_{P_{1}} + y_{P_{2}} + y_{P_{3}}).$$  
Konačno rješenje:

$$C_{1} + C_{2}e^{2x} + C_{3}e^{-2x} + \left(\frac{1}{16}x^{2} -\frac{3}{32}x \right)e^{2x} + \frac{1}{5} \cos{x} -\frac{1}{8}x^{2}. $$

\hskip 10 cm $\clubsuit$ \\

\end{frame}

\end{document}
